\documentclass[]{article}
\usepackage{lmodern}
\usepackage{amssymb,amsmath}
\usepackage{ifxetex,ifluatex}
\usepackage{fixltx2e} % provides \textsubscript
\ifnum 0\ifxetex 1\fi\ifluatex 1\fi=0 % if pdftex
  \usepackage[T1]{fontenc}
  \usepackage[utf8]{inputenc}
\else % if luatex or xelatex
  \ifxetex
    \usepackage{mathspec}
  \else
    \usepackage{fontspec}
  \fi
  \defaultfontfeatures{Ligatures=TeX,Scale=MatchLowercase}
\fi
% use upquote if available, for straight quotes in verbatim environments
\IfFileExists{upquote.sty}{\usepackage{upquote}}{}
% use microtype if available
\IfFileExists{microtype.sty}{%
\usepackage{microtype}
\UseMicrotypeSet[protrusion]{basicmath} % disable protrusion for tt fonts
}{}
\usepackage[left=0.5cm,right=0.5cm,top=0.5cm,bottom=1.5cm]{geometry}
\usepackage{hyperref}
\hypersetup{unicode=true,
            pdftitle={Traditional Graphics},
            pdfborder={0 0 0},
            breaklinks=true}
\urlstyle{same}  % don't use monospace font for urls
\usepackage{color}
\usepackage{fancyvrb}
\newcommand{\VerbBar}{|}
\newcommand{\VERB}{\Verb[commandchars=\\\{\}]}
\DefineVerbatimEnvironment{Highlighting}{Verbatim}{commandchars=\\\{\}}
% Add ',fontsize=\small' for more characters per line
\usepackage{framed}
\definecolor{shadecolor}{RGB}{248,248,248}
\newenvironment{Shaded}{\begin{snugshade}}{\end{snugshade}}
\newcommand{\KeywordTok}[1]{\textcolor[rgb]{0.13,0.29,0.53}{\textbf{#1}}}
\newcommand{\DataTypeTok}[1]{\textcolor[rgb]{0.13,0.29,0.53}{#1}}
\newcommand{\DecValTok}[1]{\textcolor[rgb]{0.00,0.00,0.81}{#1}}
\newcommand{\BaseNTok}[1]{\textcolor[rgb]{0.00,0.00,0.81}{#1}}
\newcommand{\FloatTok}[1]{\textcolor[rgb]{0.00,0.00,0.81}{#1}}
\newcommand{\ConstantTok}[1]{\textcolor[rgb]{0.00,0.00,0.00}{#1}}
\newcommand{\CharTok}[1]{\textcolor[rgb]{0.31,0.60,0.02}{#1}}
\newcommand{\SpecialCharTok}[1]{\textcolor[rgb]{0.00,0.00,0.00}{#1}}
\newcommand{\StringTok}[1]{\textcolor[rgb]{0.31,0.60,0.02}{#1}}
\newcommand{\VerbatimStringTok}[1]{\textcolor[rgb]{0.31,0.60,0.02}{#1}}
\newcommand{\SpecialStringTok}[1]{\textcolor[rgb]{0.31,0.60,0.02}{#1}}
\newcommand{\ImportTok}[1]{#1}
\newcommand{\CommentTok}[1]{\textcolor[rgb]{0.56,0.35,0.01}{\textit{#1}}}
\newcommand{\DocumentationTok}[1]{\textcolor[rgb]{0.56,0.35,0.01}{\textbf{\textit{#1}}}}
\newcommand{\AnnotationTok}[1]{\textcolor[rgb]{0.56,0.35,0.01}{\textbf{\textit{#1}}}}
\newcommand{\CommentVarTok}[1]{\textcolor[rgb]{0.56,0.35,0.01}{\textbf{\textit{#1}}}}
\newcommand{\OtherTok}[1]{\textcolor[rgb]{0.56,0.35,0.01}{#1}}
\newcommand{\FunctionTok}[1]{\textcolor[rgb]{0.00,0.00,0.00}{#1}}
\newcommand{\VariableTok}[1]{\textcolor[rgb]{0.00,0.00,0.00}{#1}}
\newcommand{\ControlFlowTok}[1]{\textcolor[rgb]{0.13,0.29,0.53}{\textbf{#1}}}
\newcommand{\OperatorTok}[1]{\textcolor[rgb]{0.81,0.36,0.00}{\textbf{#1}}}
\newcommand{\BuiltInTok}[1]{#1}
\newcommand{\ExtensionTok}[1]{#1}
\newcommand{\PreprocessorTok}[1]{\textcolor[rgb]{0.56,0.35,0.01}{\textit{#1}}}
\newcommand{\AttributeTok}[1]{\textcolor[rgb]{0.77,0.63,0.00}{#1}}
\newcommand{\RegionMarkerTok}[1]{#1}
\newcommand{\InformationTok}[1]{\textcolor[rgb]{0.56,0.35,0.01}{\textbf{\textit{#1}}}}
\newcommand{\WarningTok}[1]{\textcolor[rgb]{0.56,0.35,0.01}{\textbf{\textit{#1}}}}
\newcommand{\AlertTok}[1]{\textcolor[rgb]{0.94,0.16,0.16}{#1}}
\newcommand{\ErrorTok}[1]{\textcolor[rgb]{0.64,0.00,0.00}{\textbf{#1}}}
\newcommand{\NormalTok}[1]{#1}
\usepackage{graphicx,grffile}
\makeatletter
\def\maxwidth{\ifdim\Gin@nat@width>\linewidth\linewidth\else\Gin@nat@width\fi}
\def\maxheight{\ifdim\Gin@nat@height>\textheight\textheight\else\Gin@nat@height\fi}
\makeatother
% Scale images if necessary, so that they will not overflow the page
% margins by default, and it is still possible to overwrite the defaults
% using explicit options in \includegraphics[width, height, ...]{}
\setkeys{Gin}{width=\maxwidth,height=\maxheight,keepaspectratio}
\IfFileExists{parskip.sty}{%
\usepackage{parskip}
}{% else
\setlength{\parindent}{0pt}
\setlength{\parskip}{6pt plus 2pt minus 1pt}
}
\setlength{\emergencystretch}{3em}  % prevent overfull lines
\providecommand{\tightlist}{%
  \setlength{\itemsep}{0pt}\setlength{\parskip}{0pt}}
\setcounter{secnumdepth}{0}
% Redefines (sub)paragraphs to behave more like sections
\ifx\paragraph\undefined\else
\let\oldparagraph\paragraph
\renewcommand{\paragraph}[1]{\oldparagraph{#1}\mbox{}}
\fi
\ifx\subparagraph\undefined\else
\let\oldsubparagraph\subparagraph
\renewcommand{\subparagraph}[1]{\oldsubparagraph{#1}\mbox{}}
\fi

%%% Use protect on footnotes to avoid problems with footnotes in titles
\let\rmarkdownfootnote\footnote%
\def\footnote{\protect\rmarkdownfootnote}

%%% Change title format to be more compact
\usepackage{titling}

% Create subtitle command for use in maketitle
\newcommand{\subtitle}[1]{
  \posttitle{
    \begin{center}\large#1\end{center}
    }
}

\setlength{\droptitle}{-2em}

  \title{Traditional Graphics}
    \pretitle{\vspace{\droptitle}\centering\huge}
  \posttitle{\par}
    \author{}
    \preauthor{}\postauthor{}
    \date{}
    \predate{}\postdate{}
  

\begin{document}
\maketitle

The \texttt{plot} function can be used to make 2-d plots of (discrete)
data.

\begin{Shaded}
\begin{Highlighting}[]
\NormalTok{x <-}\StringTok{ }\KeywordTok{seq}\NormalTok{(}\DecValTok{0}\NormalTok{,}\DecValTok{1}\NormalTok{,}\FloatTok{0.1}\NormalTok{)}
\NormalTok{y <-}\StringTok{ }\NormalTok{x}\OperatorTok{^}\DecValTok{2}
\KeywordTok{plot}\NormalTok{(x, y)}
\end{Highlighting}
\end{Shaded}

\includegraphics{traditional_files/figure-latex/unnamed-chunk-1-1.pdf}

We can also connect the points with lines.

\begin{Shaded}
\begin{Highlighting}[]
\NormalTok{x <-}\StringTok{ }\KeywordTok{seq}\NormalTok{(}\DecValTok{0}\NormalTok{,}\DecValTok{1}\NormalTok{,}\FloatTok{0.1}\NormalTok{)}
\NormalTok{y <-}\StringTok{ }\NormalTok{x}\OperatorTok{^}\DecValTok{2}
\KeywordTok{plot}\NormalTok{(x, y, }\DataTypeTok{type =} \StringTok{"b"}\NormalTok{)}
\end{Highlighting}
\end{Shaded}

\includegraphics{traditional_files/figure-latex/unnamed-chunk-2-1.pdf}

We can plot a function, say the normal density function \texttt{dnorm}.

\begin{Shaded}
\begin{Highlighting}[]
\KeywordTok{curve}\NormalTok{(dnorm, }\OperatorTok{-}\DecValTok{4}\NormalTok{, }\DecValTok{4}\NormalTok{)}
\end{Highlighting}
\end{Shaded}

\includegraphics{traditional_files/figure-latex/unnamed-chunk-3-1.pdf}

Create a histogram using the included \texttt{Nile} dataset.

\begin{Shaded}
\begin{Highlighting}[]
\KeywordTok{print}\NormalTok{(Nile)}
\end{Highlighting}
\end{Shaded}

\begin{verbatim}
## Time Series:
## Start = 1871 
## End = 1970 
## Frequency = 1 
##   [1] 1120 1160  963 1210 1160 1160  813 1230 1370 1140  995  935 1110  994
##  [15] 1020  960 1180  799  958 1140 1100 1210 1150 1250 1260 1220 1030 1100
##  [29]  774  840  874  694  940  833  701  916  692 1020 1050  969  831  726
##  [43]  456  824  702 1120 1100  832  764  821  768  845  864  862  698  845
##  [57]  744  796 1040  759  781  865  845  944  984  897  822 1010  771  676
##  [71]  649  846  812  742  801 1040  860  874  848  890  744  749  838 1050
##  [85]  918  986  797  923  975  815 1020  906  901 1170  912  746  919  718
##  [99]  714  740
\end{verbatim}

\begin{Shaded}
\begin{Highlighting}[]
\KeywordTok{hist}\NormalTok{(Nile)}
\end{Highlighting}
\end{Shaded}

\includegraphics{traditional_files/figure-latex/unnamed-chunk-4-1.pdf}

Boxplots can be created very quickly. Consider the \texttt{chickwts}
dataset.

\begin{Shaded}
\begin{Highlighting}[]
\KeywordTok{head}\NormalTok{(chickwts)}
\end{Highlighting}
\end{Shaded}

\begin{verbatim}
##   weight      feed
## 1    179 horsebean
## 2    160 horsebean
## 3    136 horsebean
## 4    227 horsebean
## 5    217 horsebean
## 6    168 horsebean
\end{verbatim}

\begin{Shaded}
\begin{Highlighting}[]
\KeywordTok{boxplot}\NormalTok{(weight }\OperatorTok{~}\StringTok{ }\NormalTok{feed, }\DataTypeTok{data =}\NormalTok{ chickwts)}
\end{Highlighting}
\end{Shaded}

\includegraphics{traditional_files/figure-latex/unnamed-chunk-5-1.pdf}

Overlay a curve on a plot.

\begin{Shaded}
\begin{Highlighting}[]
\NormalTok{mu.hat <-}\StringTok{ }\KeywordTok{mean}\NormalTok{(Nile)}
\NormalTok{sigma.hat <-}\StringTok{ }\KeywordTok{sd}\NormalTok{(Nile)}
\KeywordTok{hist}\NormalTok{(Nile, }\DataTypeTok{freq =} \OtherTok{FALSE}\NormalTok{)}
\KeywordTok{curve}\NormalTok{(}\KeywordTok{dnorm}\NormalTok{(x, }\DataTypeTok{mean =}\NormalTok{ mu.hat, }\DataTypeTok{sd =}\NormalTok{ sigma.hat), }\DataTypeTok{add =} \OtherTok{TRUE}\NormalTok{)}
\end{Highlighting}
\end{Shaded}

\includegraphics{traditional_files/figure-latex/unnamed-chunk-6-1.pdf}

Instead of having the plot display in a window, request it to be written
to a file using functions like \texttt{pdf}, \texttt{jpeg}, and
\texttt{bmp}.

\begin{Shaded}
\begin{Highlighting}[]
\KeywordTok{pdf}\NormalTok{(}\StringTok{"plot.pdf"}\NormalTok{, }\DataTypeTok{height=}\DecValTok{4}\NormalTok{, }\DataTypeTok{width=}\DecValTok{4}\NormalTok{)}
\KeywordTok{hist}\NormalTok{(Nile)}
\KeywordTok{dev.off}\NormalTok{()}
\end{Highlighting}
\end{Shaded}

\begin{itemize}
\tightlist
\item
  The function \texttt{dev.off()} closes the graphics file after writing
  to it.
\item
  The units of \texttt{height} and \texttt{weight} are inches.
\item
  To save a plot during Rstudio during interactive use, see the
  \texttt{Export} menu above the plot.
\end{itemize}

There are lower-level plotting functions available for drawing shapes.
\texttt{abline} can be used to draw lines, e.g.~to plot a simple linear
regression.

\begin{Shaded}
\begin{Highlighting}[]
\KeywordTok{head}\NormalTok{(cars)}
\end{Highlighting}
\end{Shaded}

\begin{verbatim}
##   speed dist
## 1     4    2
## 2     4   10
## 3     7    4
## 4     7   22
## 5     8   16
## 6     9   10
\end{verbatim}

\begin{Shaded}
\begin{Highlighting}[]
\NormalTok{fit <-}\StringTok{ }\KeywordTok{lm}\NormalTok{(cars}\OperatorTok{$}\NormalTok{dist }\OperatorTok{~}\StringTok{ }\NormalTok{cars}\OperatorTok{$}\NormalTok{speed)}
\KeywordTok{print}\NormalTok{(fit)}
\end{Highlighting}
\end{Shaded}

\begin{verbatim}
## 
## Call:
## lm(formula = cars$dist ~ cars$speed)
## 
## Coefficients:
## (Intercept)   cars$speed  
##     -17.579        3.932
\end{verbatim}

\begin{Shaded}
\begin{Highlighting}[]
\KeywordTok{plot}\NormalTok{(cars)}
\KeywordTok{abline}\NormalTok{(}\DataTypeTok{coef =}\NormalTok{ fit}\OperatorTok{$}\NormalTok{coefficients)}
\end{Highlighting}
\end{Shaded}

\includegraphics{traditional_files/figure-latex/unnamed-chunk-8-1.pdf}

Draw polygons by connecting points.

\begin{Shaded}
\begin{Highlighting}[]
\NormalTok{x <-}\StringTok{ }\KeywordTok{c}\NormalTok{(}\DecValTok{1}\NormalTok{, }\DecValTok{0}\NormalTok{, }\OperatorTok{-}\DecValTok{1}\NormalTok{, }\DecValTok{0}\NormalTok{)}
\NormalTok{y <-}\StringTok{ }\KeywordTok{c}\NormalTok{(}\DecValTok{0}\NormalTok{, }\DecValTok{1}\NormalTok{, }\DecValTok{0}\NormalTok{, }\OperatorTok{-}\DecValTok{1}\NormalTok{)}
\KeywordTok{plot}\NormalTok{(x,y)}
\KeywordTok{polygon}\NormalTok{(x,y)}
\end{Highlighting}
\end{Shaded}

\includegraphics{traditional_files/figure-latex/unnamed-chunk-9-1.pdf}

Draw rectangles with \texttt{rect} and line segments with
\texttt{segments}.

\begin{Shaded}
\begin{Highlighting}[]
\KeywordTok{plot.new}\NormalTok{()  }\CommentTok{# Create an empty plot window, and add the parts manually}
\KeywordTok{plot.window}\NormalTok{(}\DataTypeTok{xlim=}\KeywordTok{c}\NormalTok{(}\OperatorTok{-}\DecValTok{2}\NormalTok{,}\DecValTok{2}\NormalTok{), }\DataTypeTok{ylim=}\KeywordTok{c}\NormalTok{(}\OperatorTok{-}\DecValTok{2}\NormalTok{,}\DecValTok{2}\NormalTok{))}
\KeywordTok{axis}\NormalTok{(}\DecValTok{1}\NormalTok{)}
\KeywordTok{axis}\NormalTok{(}\DecValTok{2}\NormalTok{)}
\KeywordTok{title}\NormalTok{(}\DataTypeTok{main=}\StringTok{"My title"}\NormalTok{)}
\KeywordTok{title}\NormalTok{(}\DataTypeTok{xlab=}\StringTok{"x-label"}\NormalTok{)}
\KeywordTok{title}\NormalTok{(}\DataTypeTok{ylab=}\StringTok{"y-label"}\NormalTok{)}
\KeywordTok{box}\NormalTok{()}
\KeywordTok{rect}\NormalTok{(}\DataTypeTok{xleft =} \OperatorTok{-}\DecValTok{1}\NormalTok{, }\DataTypeTok{ybottom =} \OperatorTok{-}\DecValTok{1}\NormalTok{, }\DataTypeTok{xright =} \DecValTok{1}\NormalTok{, }\DataTypeTok{ytop =} \DecValTok{1}\NormalTok{, }\DataTypeTok{col=}\StringTok{"yellow"}\NormalTok{, }\DataTypeTok{border =} \StringTok{"red"}\NormalTok{)}
\KeywordTok{segments}\NormalTok{(}\DataTypeTok{x0 =} \OperatorTok{-}\DecValTok{2}\NormalTok{, }\DataTypeTok{y0 =} \OperatorTok{-}\DecValTok{2}\NormalTok{, }\DataTypeTok{x1 =} \DecValTok{0}\NormalTok{, }\DataTypeTok{y1 =} \DecValTok{2}\NormalTok{, }\DataTypeTok{col=}\StringTok{"green"}\NormalTok{)}
\KeywordTok{segments}\NormalTok{(}\DataTypeTok{x0 =} \DecValTok{0}\NormalTok{, }\DataTypeTok{y0 =} \DecValTok{2}\NormalTok{, }\DataTypeTok{x1 =} \DecValTok{2}\NormalTok{, }\DataTypeTok{y1 =} \OperatorTok{-}\DecValTok{2}\NormalTok{, }\DataTypeTok{col=}\StringTok{"green"}\NormalTok{)}
\KeywordTok{segments}\NormalTok{(}\DataTypeTok{x0 =} \OperatorTok{-}\DecValTok{2}\NormalTok{, }\DataTypeTok{y0 =} \OperatorTok{-}\DecValTok{2}\NormalTok{, }\DataTypeTok{x1 =} \DecValTok{2}\NormalTok{, }\DataTypeTok{y1 =} \OperatorTok{-}\DecValTok{2}\NormalTok{, }\DataTypeTok{col=}\StringTok{"green"}\NormalTok{)}
\end{Highlighting}
\end{Shaded}

\includegraphics{traditional_files/figure-latex/unnamed-chunk-10-1.pdf}

Multiple plots in one figure.

\begin{Shaded}
\begin{Highlighting}[]
\KeywordTok{par}\NormalTok{(}\DataTypeTok{mfrow =} \KeywordTok{c}\NormalTok{(}\DecValTok{1}\NormalTok{,}\DecValTok{2}\NormalTok{))}
\KeywordTok{curve}\NormalTok{(}\KeywordTok{dnorm}\NormalTok{(x), }\DataTypeTok{xlim =} \KeywordTok{c}\NormalTok{(}\OperatorTok{-}\DecValTok{3}\NormalTok{,}\DecValTok{3}\NormalTok{))}
\KeywordTok{curve}\NormalTok{(}\KeywordTok{pnorm}\NormalTok{(x), }\DataTypeTok{xlim =} \KeywordTok{c}\NormalTok{(}\OperatorTok{-}\DecValTok{3}\NormalTok{,}\DecValTok{3}\NormalTok{))}
\end{Highlighting}
\end{Shaded}

\includegraphics{traditional_files/figure-latex/unnamed-chunk-11-1.pdf}

\begin{Shaded}
\begin{Highlighting}[]
\KeywordTok{dev.off}\NormalTok{()}
\end{Highlighting}
\end{Shaded}

\begin{verbatim}
## null device 
##           1
\end{verbatim}


\end{document}
