\documentclass[]{article}
\usepackage{lmodern}
\usepackage{amssymb,amsmath}
\usepackage{ifxetex,ifluatex}
\usepackage{fixltx2e} % provides \textsubscript
\ifnum 0\ifxetex 1\fi\ifluatex 1\fi=0 % if pdftex
  \usepackage[T1]{fontenc}
  \usepackage[utf8]{inputenc}
\else % if luatex or xelatex
  \ifxetex
    \usepackage{mathspec}
  \else
    \usepackage{fontspec}
  \fi
  \defaultfontfeatures{Ligatures=TeX,Scale=MatchLowercase}
\fi
% use upquote if available, for straight quotes in verbatim environments
\IfFileExists{upquote.sty}{\usepackage{upquote}}{}
% use microtype if available
\IfFileExists{microtype.sty}{%
\usepackage{microtype}
\UseMicrotypeSet[protrusion]{basicmath} % disable protrusion for tt fonts
}{}
\usepackage[left=0.5cm,right=0.5cm,top=0.5cm,bottom=1.5cm]{geometry}
\usepackage{hyperref}
\hypersetup{unicode=true,
            pdftitle={Graphics with ggplot2},
            pdfborder={0 0 0},
            breaklinks=true}
\urlstyle{same}  % don't use monospace font for urls
\usepackage{color}
\usepackage{fancyvrb}
\newcommand{\VerbBar}{|}
\newcommand{\VERB}{\Verb[commandchars=\\\{\}]}
\DefineVerbatimEnvironment{Highlighting}{Verbatim}{commandchars=\\\{\}}
% Add ',fontsize=\small' for more characters per line
\usepackage{framed}
\definecolor{shadecolor}{RGB}{248,248,248}
\newenvironment{Shaded}{\begin{snugshade}}{\end{snugshade}}
\newcommand{\KeywordTok}[1]{\textcolor[rgb]{0.13,0.29,0.53}{\textbf{#1}}}
\newcommand{\DataTypeTok}[1]{\textcolor[rgb]{0.13,0.29,0.53}{#1}}
\newcommand{\DecValTok}[1]{\textcolor[rgb]{0.00,0.00,0.81}{#1}}
\newcommand{\BaseNTok}[1]{\textcolor[rgb]{0.00,0.00,0.81}{#1}}
\newcommand{\FloatTok}[1]{\textcolor[rgb]{0.00,0.00,0.81}{#1}}
\newcommand{\ConstantTok}[1]{\textcolor[rgb]{0.00,0.00,0.00}{#1}}
\newcommand{\CharTok}[1]{\textcolor[rgb]{0.31,0.60,0.02}{#1}}
\newcommand{\SpecialCharTok}[1]{\textcolor[rgb]{0.00,0.00,0.00}{#1}}
\newcommand{\StringTok}[1]{\textcolor[rgb]{0.31,0.60,0.02}{#1}}
\newcommand{\VerbatimStringTok}[1]{\textcolor[rgb]{0.31,0.60,0.02}{#1}}
\newcommand{\SpecialStringTok}[1]{\textcolor[rgb]{0.31,0.60,0.02}{#1}}
\newcommand{\ImportTok}[1]{#1}
\newcommand{\CommentTok}[1]{\textcolor[rgb]{0.56,0.35,0.01}{\textit{#1}}}
\newcommand{\DocumentationTok}[1]{\textcolor[rgb]{0.56,0.35,0.01}{\textbf{\textit{#1}}}}
\newcommand{\AnnotationTok}[1]{\textcolor[rgb]{0.56,0.35,0.01}{\textbf{\textit{#1}}}}
\newcommand{\CommentVarTok}[1]{\textcolor[rgb]{0.56,0.35,0.01}{\textbf{\textit{#1}}}}
\newcommand{\OtherTok}[1]{\textcolor[rgb]{0.56,0.35,0.01}{#1}}
\newcommand{\FunctionTok}[1]{\textcolor[rgb]{0.00,0.00,0.00}{#1}}
\newcommand{\VariableTok}[1]{\textcolor[rgb]{0.00,0.00,0.00}{#1}}
\newcommand{\ControlFlowTok}[1]{\textcolor[rgb]{0.13,0.29,0.53}{\textbf{#1}}}
\newcommand{\OperatorTok}[1]{\textcolor[rgb]{0.81,0.36,0.00}{\textbf{#1}}}
\newcommand{\BuiltInTok}[1]{#1}
\newcommand{\ExtensionTok}[1]{#1}
\newcommand{\PreprocessorTok}[1]{\textcolor[rgb]{0.56,0.35,0.01}{\textit{#1}}}
\newcommand{\AttributeTok}[1]{\textcolor[rgb]{0.77,0.63,0.00}{#1}}
\newcommand{\RegionMarkerTok}[1]{#1}
\newcommand{\InformationTok}[1]{\textcolor[rgb]{0.56,0.35,0.01}{\textbf{\textit{#1}}}}
\newcommand{\WarningTok}[1]{\textcolor[rgb]{0.56,0.35,0.01}{\textbf{\textit{#1}}}}
\newcommand{\AlertTok}[1]{\textcolor[rgb]{0.94,0.16,0.16}{#1}}
\newcommand{\ErrorTok}[1]{\textcolor[rgb]{0.64,0.00,0.00}{\textbf{#1}}}
\newcommand{\NormalTok}[1]{#1}
\usepackage{graphicx,grffile}
\makeatletter
\def\maxwidth{\ifdim\Gin@nat@width>\linewidth\linewidth\else\Gin@nat@width\fi}
\def\maxheight{\ifdim\Gin@nat@height>\textheight\textheight\else\Gin@nat@height\fi}
\makeatother
% Scale images if necessary, so that they will not overflow the page
% margins by default, and it is still possible to overwrite the defaults
% using explicit options in \includegraphics[width, height, ...]{}
\setkeys{Gin}{width=\maxwidth,height=\maxheight,keepaspectratio}
\IfFileExists{parskip.sty}{%
\usepackage{parskip}
}{% else
\setlength{\parindent}{0pt}
\setlength{\parskip}{6pt plus 2pt minus 1pt}
}
\setlength{\emergencystretch}{3em}  % prevent overfull lines
\providecommand{\tightlist}{%
  \setlength{\itemsep}{0pt}\setlength{\parskip}{0pt}}
\setcounter{secnumdepth}{5}
% Redefines (sub)paragraphs to behave more like sections
\ifx\paragraph\undefined\else
\let\oldparagraph\paragraph
\renewcommand{\paragraph}[1]{\oldparagraph{#1}\mbox{}}
\fi
\ifx\subparagraph\undefined\else
\let\oldsubparagraph\subparagraph
\renewcommand{\subparagraph}[1]{\oldsubparagraph{#1}\mbox{}}
\fi

%%% Use protect on footnotes to avoid problems with footnotes in titles
\let\rmarkdownfootnote\footnote%
\def\footnote{\protect\rmarkdownfootnote}

%%% Change title format to be more compact
\usepackage{titling}

% Create subtitle command for use in maketitle
\newcommand{\subtitle}[1]{
  \posttitle{
    \begin{center}\large#1\end{center}
    }
}

\setlength{\droptitle}{-2em}

  \title{Graphics with ggplot2}
    \pretitle{\vspace{\droptitle}\centering\huge}
  \posttitle{\par}
    \author{}
    \preauthor{}\postauthor{}
    \date{}
    \predate{}\postdate{}
  

\begin{document}
\maketitle

\begin{Shaded}
\begin{Highlighting}[]
\KeywordTok{library}\NormalTok{(ggplot2)}
\end{Highlighting}
\end{Shaded}

Consider the \texttt{mpg} dataset.

\begin{Shaded}
\begin{Highlighting}[]
\KeywordTok{tail}\NormalTok{(mpg)}
\end{Highlighting}
\end{Shaded}

\begin{verbatim}
## # A tibble: 6 x 11
##   manufacturer model displ  year   cyl trans drv     cty   hwy fl    class
##   <chr>        <chr> <dbl> <int> <int> <chr> <chr> <int> <int> <chr> <chr>
## 1 volkswagen   pass~   1.8  1999     4 auto~ f        18    29 p     mids~
## 2 volkswagen   pass~   2    2008     4 auto~ f        19    28 p     mids~
## 3 volkswagen   pass~   2    2008     4 manu~ f        21    29 p     mids~
## 4 volkswagen   pass~   2.8  1999     6 auto~ f        16    26 p     mids~
## 5 volkswagen   pass~   2.8  1999     6 manu~ f        18    26 p     mids~
## 6 volkswagen   pass~   3.6  2008     6 auto~ f        17    26 p     mids~
\end{verbatim}

Make an empty plot with the data prepared.

\begin{Shaded}
\begin{Highlighting}[]
\NormalTok{g <-}\StringTok{ }\KeywordTok{ggplot}\NormalTok{(mpg, }\KeywordTok{aes}\NormalTok{(cty, hwy))}
\KeywordTok{print}\NormalTok{(g)}
\end{Highlighting}
\end{Shaded}

\includegraphics{ggplot2_files/figure-latex/unnamed-chunk-3-1.pdf}

Actually plot some points.

\begin{Shaded}
\begin{Highlighting}[]
\NormalTok{g <-}\StringTok{ }\KeywordTok{ggplot}\NormalTok{(mpg, }\KeywordTok{aes}\NormalTok{(cty, hwy)) }\OperatorTok{+}
\StringTok{    }\KeywordTok{geom_point}\NormalTok{()}
\KeywordTok{print}\NormalTok{(g)}
\end{Highlighting}
\end{Shaded}

\includegraphics{ggplot2_files/figure-latex/unnamed-chunk-4-1.pdf}

Add a loess smoother.

\begin{Shaded}
\begin{Highlighting}[]
\NormalTok{g <-}\StringTok{ }\KeywordTok{ggplot}\NormalTok{(mpg, }\KeywordTok{aes}\NormalTok{(cty, hwy)) }\OperatorTok{+}
\StringTok{    }\KeywordTok{geom_point}\NormalTok{() }\OperatorTok{+}
\StringTok{    }\KeywordTok{geom_smooth}\NormalTok{()}
\KeywordTok{print}\NormalTok{(g)}
\end{Highlighting}
\end{Shaded}

\begin{verbatim}
## `geom_smooth()` using method = 'loess' and formula 'y ~ x'
\end{verbatim}

\includegraphics{ggplot2_files/figure-latex/unnamed-chunk-5-1.pdf}

Customize axes, theme, etc.

\begin{Shaded}
\begin{Highlighting}[]
\NormalTok{g <-}\StringTok{ }\KeywordTok{ggplot}\NormalTok{(mpg, }\KeywordTok{aes}\NormalTok{(cty, hwy)) }\OperatorTok{+}
\StringTok{    }\KeywordTok{geom_point}\NormalTok{(}\DataTypeTok{shape =} \DecValTok{17}\NormalTok{, }\DataTypeTok{size =} \DecValTok{2}\NormalTok{) }\OperatorTok{+}
\StringTok{    }\KeywordTok{geom_smooth}\NormalTok{() }\OperatorTok{+}
\StringTok{    }\KeywordTok{labs}\NormalTok{(}\DataTypeTok{x =} \StringTok{"City"}\NormalTok{, }\DataTypeTok{y =} \StringTok{"Highway"}\NormalTok{, }\DataTypeTok{title =} \StringTok{"Miles Per Gallon"}\NormalTok{) }\OperatorTok{+}
\StringTok{    }\KeywordTok{theme_bw}\NormalTok{() }\OperatorTok{+}
\StringTok{    }\KeywordTok{theme}\NormalTok{(}\DataTypeTok{panel.grid.major =} \KeywordTok{element_blank}\NormalTok{(),}
          \DataTypeTok{panel.grid.minor =} \KeywordTok{element_blank}\NormalTok{())}
\KeywordTok{print}\NormalTok{(g)}
\end{Highlighting}
\end{Shaded}

\begin{verbatim}
## `geom_smooth()` using method = 'loess' and formula 'y ~ x'
\end{verbatim}

\includegraphics{ggplot2_files/figure-latex/unnamed-chunk-6-1.pdf}

Now plot a histogram.

\begin{Shaded}
\begin{Highlighting}[]
\NormalTok{g <-}\StringTok{ }\KeywordTok{ggplot}\NormalTok{(mpg, }\KeywordTok{aes}\NormalTok{(hwy)) }\OperatorTok{+}
\StringTok{    }\KeywordTok{geom_histogram}\NormalTok{()}
\KeywordTok{print}\NormalTok{(g)}
\end{Highlighting}
\end{Shaded}

\begin{verbatim}
## `stat_bin()` using `bins = 30`. Pick better value with `binwidth`.
\end{verbatim}

\includegraphics{ggplot2_files/figure-latex/unnamed-chunk-7-1.pdf}

Customize the histogram.

\begin{Shaded}
\begin{Highlighting}[]
\NormalTok{g <-}\StringTok{ }\KeywordTok{ggplot}\NormalTok{(mpg, }\KeywordTok{aes}\NormalTok{(hwy)) }\OperatorTok{+}
\StringTok{    }\KeywordTok{geom_histogram}\NormalTok{(}\DataTypeTok{bins=}\DecValTok{20}\NormalTok{, }\DataTypeTok{fill=}\StringTok{"blue"}\NormalTok{, }\DataTypeTok{color=}\StringTok{"black"}\NormalTok{, }\DataTypeTok{size=}\FloatTok{0.25}\NormalTok{)}
\KeywordTok{print}\NormalTok{(g)}
\end{Highlighting}
\end{Shaded}

\includegraphics{ggplot2_files/figure-latex/unnamed-chunk-8-1.pdf}

Plot a histogram for each value of \texttt{cyl}.

\begin{Shaded}
\begin{Highlighting}[]
\NormalTok{g <-}\StringTok{ }\KeywordTok{ggplot}\NormalTok{(mpg, }\KeywordTok{aes}\NormalTok{(hwy)) }\OperatorTok{+}
\StringTok{    }\KeywordTok{geom_histogram}\NormalTok{(}\DataTypeTok{bins=}\DecValTok{20}\NormalTok{, }\DataTypeTok{fill=}\StringTok{"blue"}\NormalTok{, }\DataTypeTok{color=}\StringTok{"black"}\NormalTok{, }\DataTypeTok{size=}\FloatTok{0.25}\NormalTok{) }\OperatorTok{+}
\StringTok{    }\KeywordTok{facet_wrap}\NormalTok{(}\OperatorTok{~}\StringTok{ }\NormalTok{cyl)}
\KeywordTok{print}\NormalTok{(g)}
\end{Highlighting}
\end{Shaded}

\includegraphics{ggplot2_files/figure-latex/unnamed-chunk-9-1.pdf}

Plot boxplots for each value of \texttt{class}.

\begin{Shaded}
\begin{Highlighting}[]
\NormalTok{g <-}\StringTok{ }\KeywordTok{ggplot}\NormalTok{(mpg, }\KeywordTok{aes}\NormalTok{(class, hwy)) }\OperatorTok{+}
\StringTok{    }\KeywordTok{geom_boxplot}\NormalTok{()}
\KeywordTok{print}\NormalTok{(g)}
\end{Highlighting}
\end{Shaded}

\includegraphics{ggplot2_files/figure-latex/unnamed-chunk-10-1.pdf}

Plot two boxplots together horizontally.

\begin{Shaded}
\begin{Highlighting}[]
\KeywordTok{library}\NormalTok{(gridExtra)}

\NormalTok{g <-}\StringTok{ }\KeywordTok{ggplot}\NormalTok{(mpg, }\KeywordTok{aes}\NormalTok{(class, hwy)) }\OperatorTok{+}
\StringTok{    }\KeywordTok{geom_boxplot}\NormalTok{()}
\NormalTok{h <-}\StringTok{ }\KeywordTok{ggplot}\NormalTok{(mpg, }\KeywordTok{aes}\NormalTok{(class, cty)) }\OperatorTok{+}
\StringTok{    }\KeywordTok{geom_boxplot}\NormalTok{()}
\KeywordTok{grid.arrange}\NormalTok{(g, h)}
\end{Highlighting}
\end{Shaded}

\includegraphics{ggplot2_files/figure-latex/unnamed-chunk-11-1.pdf}

Recall the AR(2) time series model

\begin{align*}
y_t = \phi_1 y_{t-1} + \phi_2 y_{t-2} + \epsilon_t, \quad
\epsilon_t \stackrel{\text{iid}}{\sim} \text{N}(0, \sigma^2),
\quad t = 1, \ldots, n.
\end{align*}

Generate an AR(2) series and plot it.

\begin{Shaded}
\begin{Highlighting}[]
\NormalTok{y <-}\StringTok{ }\KeywordTok{arima.sim}\NormalTok{(}\DataTypeTok{n =} \DecValTok{200}\NormalTok{, }\KeywordTok{list}\NormalTok{(}\DataTypeTok{ar =} \KeywordTok{c}\NormalTok{(}\FloatTok{0.5}\NormalTok{, }\OperatorTok{-}\FloatTok{0.2}\NormalTok{), }\DataTypeTok{sd =} \KeywordTok{sqrt}\NormalTok{(}\FloatTok{0.25}\NormalTok{)))}
\NormalTok{dat <-}\StringTok{ }\KeywordTok{data.frame}\NormalTok{(}\DataTypeTok{t =} \DecValTok{1}\OperatorTok{:}\DecValTok{200}\NormalTok{, }\DataTypeTok{y =} \KeywordTok{as.numeric}\NormalTok{(y))}
\NormalTok{g <-}\StringTok{ }\KeywordTok{ggplot}\NormalTok{(dat, }\KeywordTok{aes}\NormalTok{(t, y)) }\OperatorTok{+}\StringTok{ }\KeywordTok{geom_line}\NormalTok{()}
\KeywordTok{print}\NormalTok{(g)}
\end{Highlighting}
\end{Shaded}

\includegraphics{ggplot2_files/figure-latex/unnamed-chunk-12-1.pdf}

What about plotting multiple series on one plot? First draw the series.

\begin{Shaded}
\begin{Highlighting}[]
\NormalTok{n <-}\StringTok{ }\DecValTok{200}
\NormalTok{y1 <-}\StringTok{ }\DecValTok{0} \OperatorTok{+}\StringTok{ }\KeywordTok{arima.sim}\NormalTok{(}\DataTypeTok{n =}\NormalTok{ n, }\KeywordTok{list}\NormalTok{(}\DataTypeTok{ar =} \KeywordTok{c}\NormalTok{(}\FloatTok{0.5}\NormalTok{, }\OperatorTok{-}\FloatTok{0.2}\NormalTok{), }\DataTypeTok{sd =} \KeywordTok{sqrt}\NormalTok{(}\FloatTok{0.25}\NormalTok{)))}
\NormalTok{y2 <-}\StringTok{ }\DecValTok{3} \OperatorTok{+}\StringTok{ }\KeywordTok{arima.sim}\NormalTok{(}\DataTypeTok{n =}\NormalTok{ n, }\KeywordTok{list}\NormalTok{(}\DataTypeTok{ar =} \KeywordTok{c}\NormalTok{(}\FloatTok{0.1}\NormalTok{, }\OperatorTok{-}\FloatTok{0.2}\NormalTok{), }\DataTypeTok{sd =} \KeywordTok{sqrt}\NormalTok{(}\FloatTok{0.25}\NormalTok{)))}
\NormalTok{y3 <-}\StringTok{ }\OperatorTok{-}\DecValTok{3} \OperatorTok{+}\StringTok{ }\KeywordTok{arima.sim}\NormalTok{(}\DataTypeTok{n =}\NormalTok{ n, }\KeywordTok{list}\NormalTok{(}\DataTypeTok{ar =} \KeywordTok{c}\NormalTok{(}\FloatTok{0.7}\NormalTok{, }\OperatorTok{-}\FloatTok{0.2}\NormalTok{), }\DataTypeTok{sd =} \KeywordTok{sqrt}\NormalTok{(}\FloatTok{0.5}\NormalTok{)))}
\end{Highlighting}
\end{Shaded}

Make the series columns of a \texttt{data.frame}.

\begin{Shaded}
\begin{Highlighting}[]
\NormalTok{dat <-}\StringTok{ }\KeywordTok{data.frame}\NormalTok{(}\DataTypeTok{t =} \DecValTok{1}\OperatorTok{:}\NormalTok{n, }\DataTypeTok{y1 =} \KeywordTok{as.numeric}\NormalTok{(y1), }\DataTypeTok{y2 =} \KeywordTok{as.numeric}\NormalTok{(y2), }\DataTypeTok{y3 =} \KeywordTok{as.numeric}\NormalTok{(y3))}
\KeywordTok{head}\NormalTok{(dat, }\DecValTok{3}\NormalTok{)}
\end{Highlighting}
\end{Shaded}

\begin{verbatim}
##   t         y1       y2        y3
## 1 1  0.8011972 2.542057 -2.486387
## 2 2 -0.5484303 3.681503 -4.346941
## 3 3 -0.9012958 2.268507 -4.117533
\end{verbatim}

Reshape the \texttt{data.frame} by stacking the series vertically.

\begin{Shaded}
\begin{Highlighting}[]
\KeywordTok{library}\NormalTok{(reshape2)}
\NormalTok{newdat <-}\StringTok{ }\KeywordTok{melt}\NormalTok{(dat, }\StringTok{'t'}\NormalTok{)}
\KeywordTok{head}\NormalTok{(newdat)}
\end{Highlighting}
\end{Shaded}

\begin{verbatim}
##   t variable      value
## 1 1       y1  0.8011972
## 2 2       y1 -0.5484303
## 3 3       y1 -0.9012958
## 4 4       y1 -2.1567229
## 5 5       y1  0.3709560
## 6 6       y1  0.2081684
\end{verbatim}

\begin{Shaded}
\begin{Highlighting}[]
\KeywordTok{tail}\NormalTok{(newdat)}
\end{Highlighting}
\end{Shaded}

\begin{verbatim}
##       t variable     value
## 595 195       y3 -1.761622
## 596 196       y3 -1.446483
## 597 197       y3 -3.049945
## 598 198       y3 -2.259114
## 599 199       y3 -1.252249
## 600 200       y3 -2.780923
\end{verbatim}

Here is one way to plot the series together.

\begin{Shaded}
\begin{Highlighting}[]
\NormalTok{g <-}\StringTok{ }\KeywordTok{ggplot}\NormalTok{(newdat, }\KeywordTok{aes}\NormalTok{(}\DataTypeTok{x =}\NormalTok{ t, }\DataTypeTok{y =}\NormalTok{ value,}
            \DataTypeTok{group =}\NormalTok{ variable,}
            \DataTypeTok{color =}\NormalTok{ variable,}
            \DataTypeTok{linetype =}\NormalTok{ variable)) }\OperatorTok{+}
\StringTok{    }\KeywordTok{geom_line}\NormalTok{() }\OperatorTok{+}
\StringTok{    }\KeywordTok{theme}\NormalTok{(}\DataTypeTok{legend.position =} \StringTok{"bottom"}\NormalTok{) }\OperatorTok{+}
\StringTok{    }\KeywordTok{ggtitle}\NormalTok{(}\StringTok{"Multiple Series"}\NormalTok{)}
\KeywordTok{print}\NormalTok{(g)}
\end{Highlighting}
\end{Shaded}

\includegraphics{ggplot2_files/figure-latex/unnamed-chunk-16-1.pdf}

Draw from bivariate normal.

\begin{Shaded}
\begin{Highlighting}[]
\KeywordTok{library}\NormalTok{(mvtnorm)}
\NormalTok{Sigma <-}\StringTok{ }\KeywordTok{matrix}\NormalTok{(}\KeywordTok{c}\NormalTok{(}\DecValTok{1}\NormalTok{, }\DecValTok{1}\OperatorTok{/}\DecValTok{2}\NormalTok{, }\DecValTok{1}\OperatorTok{/}\DecValTok{2}\NormalTok{, }\DecValTok{1}\NormalTok{), }\DecValTok{2}\NormalTok{, }\DecValTok{2}\NormalTok{)}
\NormalTok{x <-}\StringTok{ }\KeywordTok{rmvnorm}\NormalTok{(}\DataTypeTok{n =} \DecValTok{10000}\NormalTok{, }\DataTypeTok{mean =} \KeywordTok{c}\NormalTok{(}\DecValTok{0}\NormalTok{,}\DecValTok{0}\NormalTok{), }\DataTypeTok{sigma =}\NormalTok{ Sigma)}
\NormalTok{dat <-}\StringTok{ }\KeywordTok{data.frame}\NormalTok{(x)}
\KeywordTok{colnames}\NormalTok{(dat) <-}\StringTok{ }\KeywordTok{c}\NormalTok{(}\StringTok{"x"}\NormalTok{, }\StringTok{"y"}\NormalTok{)}
\end{Highlighting}
\end{Shaded}

Plot the points and superimpose contours.

\begin{Shaded}
\begin{Highlighting}[]
\NormalTok{g <-}\StringTok{ }\KeywordTok{ggplot}\NormalTok{(dat, }\KeywordTok{aes}\NormalTok{(}\DataTypeTok{x=}\NormalTok{x, }\DataTypeTok{y=}\NormalTok{y)) }\OperatorTok{+}
\StringTok{    }\KeywordTok{geom_point}\NormalTok{() }\OperatorTok{+}
\StringTok{    }\KeywordTok{geom_density2d}\NormalTok{()}
\KeywordTok{print}\NormalTok{(g)}
\end{Highlighting}
\end{Shaded}

\includegraphics{ggplot2_files/figure-latex/unnamed-chunk-18-1.pdf}

Plot bins instead to display density values.

\begin{Shaded}
\begin{Highlighting}[]
\NormalTok{g <-}\StringTok{ }\KeywordTok{ggplot}\NormalTok{(dat, }\KeywordTok{aes}\NormalTok{(}\DataTypeTok{x=}\NormalTok{x, }\DataTypeTok{y=}\NormalTok{y)) }\OperatorTok{+}
\StringTok{    }\KeywordTok{geom_bin2d}\NormalTok{() }\OperatorTok{+}
\StringTok{    }\KeywordTok{scale_fill_gradient}\NormalTok{(}\DataTypeTok{low =} \StringTok{"blue"}\NormalTok{, }\DataTypeTok{high =} \StringTok{"red"}\NormalTok{)}
\KeywordTok{print}\NormalTok{(g)}
\end{Highlighting}
\end{Shaded}

\includegraphics{ggplot2_files/figure-latex/unnamed-chunk-19-1.pdf}

Plot hexigonal bins instead.

\begin{Shaded}
\begin{Highlighting}[]
\KeywordTok{library}\NormalTok{(hexbin)}
\NormalTok{g <-}\StringTok{ }\KeywordTok{ggplot}\NormalTok{(dat, }\KeywordTok{aes}\NormalTok{(}\DataTypeTok{x=}\NormalTok{x, }\DataTypeTok{y=}\NormalTok{y)) }\OperatorTok{+}
\StringTok{    }\KeywordTok{geom_hex}\NormalTok{() }\OperatorTok{+}
\StringTok{    }\KeywordTok{scale_fill_gradient}\NormalTok{(}\DataTypeTok{low =} \StringTok{"blue"}\NormalTok{, }\DataTypeTok{high =} \StringTok{"red"}\NormalTok{) }\OperatorTok{+}
\StringTok{    }\KeywordTok{coord_cartesian}\NormalTok{(}\DataTypeTok{xlim =} \KeywordTok{c}\NormalTok{(}\OperatorTok{-}\FloatTok{2.5}\NormalTok{, }\FloatTok{2.5}\NormalTok{), }\DataTypeTok{ylim =} \KeywordTok{c}\NormalTok{(}\OperatorTok{-}\FloatTok{2.5}\NormalTok{, }\FloatTok{2.5}\NormalTok{)) }\OperatorTok{+}
\StringTok{    }\KeywordTok{theme}\NormalTok{(}\DataTypeTok{legend.position =} \StringTok{"bottom"}\NormalTok{)}
\KeywordTok{print}\NormalTok{(g)}
\end{Highlighting}
\end{Shaded}

\includegraphics{ggplot2_files/figure-latex/unnamed-chunk-20-1.pdf}

Generate data from a trinomial distribution.

\begin{Shaded}
\begin{Highlighting}[]
\NormalTok{m <-}\StringTok{ }\DecValTok{10}
\NormalTok{grid <-}\StringTok{ }\KeywordTok{expand.grid}\NormalTok{(}\DataTypeTok{x =} \DecValTok{0}\OperatorTok{:}\NormalTok{m, }\DataTypeTok{y =} \DecValTok{0}\OperatorTok{:}\NormalTok{m)}
\NormalTok{grid <-}\StringTok{ }\NormalTok{grid[grid}\OperatorTok{$}\NormalTok{x }\OperatorTok{+}\StringTok{ }\NormalTok{grid}\OperatorTok{$}\NormalTok{y }\OperatorTok{<=}\StringTok{ }\NormalTok{m,]}
\NormalTok{grid}\OperatorTok{$}\NormalTok{z <-}\StringTok{ }\NormalTok{m }\OperatorTok{-}\StringTok{ }\NormalTok{grid}\OperatorTok{$}\NormalTok{x }\OperatorTok{-}\StringTok{ }\NormalTok{grid}\OperatorTok{$}\NormalTok{y}
\end{Highlighting}
\end{Shaded}

Plot the trinomial data on a 2-d grid (no need to display the redundant
third coordinate).

\begin{Shaded}
\begin{Highlighting}[]
\NormalTok{grid}\OperatorTok{$}\NormalTok{dens <-}\StringTok{ }\KeywordTok{apply}\NormalTok{(grid, }\DecValTok{1}\NormalTok{, dmultinom, }\DataTypeTok{size=}\NormalTok{m, }\DataTypeTok{prob=}\KeywordTok{c}\NormalTok{(}\FloatTok{0.1}\NormalTok{, }\FloatTok{0.3}\NormalTok{, }\FloatTok{0.6}\NormalTok{))}
\NormalTok{g <-}\StringTok{ }\KeywordTok{ggplot}\NormalTok{(grid, }\KeywordTok{aes}\NormalTok{(}\DataTypeTok{x=}\NormalTok{x, }\DataTypeTok{y=}\NormalTok{y)) }\OperatorTok{+}
\StringTok{    }\KeywordTok{geom_raster}\NormalTok{(}\KeywordTok{aes}\NormalTok{(}\DataTypeTok{fill =}\NormalTok{ dens)) }\OperatorTok{+}
\StringTok{    }\KeywordTok{scale_fill_gradient}\NormalTok{(}\DataTypeTok{low =} \StringTok{"yellow"}\NormalTok{, }\DataTypeTok{high =} \StringTok{"red"}\NormalTok{) }\OperatorTok{+}
\StringTok{    }\KeywordTok{scale_x_discrete}\NormalTok{(}\DataTypeTok{limits =} \DecValTok{0}\OperatorTok{:}\NormalTok{m) }\OperatorTok{+}
\StringTok{    }\KeywordTok{scale_y_discrete}\NormalTok{(}\DataTypeTok{limits =} \DecValTok{0}\OperatorTok{:}\NormalTok{m)}
\KeywordTok{print}\NormalTok{(g)}
\end{Highlighting}
\end{Shaded}

\includegraphics{ggplot2_files/figure-latex/unnamed-chunk-22-1.pdf}

To save the last plot, use \texttt{ggsave}. File type is determined by
extension (pdf, png, jpg, etc).

\begin{Shaded}
\begin{Highlighting}[]
\KeywordTok{ggsave}\NormalTok{(}\StringTok{"plot.pdf"}\NormalTok{, }\DataTypeTok{width =} \DecValTok{5}\NormalTok{, }\DataTypeTok{height =} \DecValTok{5}\NormalTok{)}
\end{Highlighting}
\end{Shaded}


\end{document}
